\documentclass[12pt, a4paper, oneside]{article}
\usepackage[utf8]{inputenc}
\usepackage[T1]{fontenc}
\usepackage[english]{babel}
\usepackage[left=2cm,right=2cm,top=2cm,bottom=2cm]{geometry}
\usepackage{listings}
\usepackage{placeins}

\usepackage{pgfplots,pgfplotstable}
\pgfplotsset{compat=1.11,
    /pgfplots/ybar legend/.style={
    /pgfplots/legend image code/.code={%
       \draw[##1,/tikz/.cd,yshift=-0.25em]
        (0cm,0cm) rectangle (3pt,0.8em);},
   },
}

\usepackage[capposition=top]{floatrow}

\usepackage{xcolor}
\usepackage{varioref}
\usepackage{hyperref}
\usepackage{cleveref}

\title{AIND: Planning Search -- Heuristics Analysis}
\author{Markus Mayer}
\date{\today}

\definecolor{mygreen}{rgb}{0,0.6,0}
\definecolor{mygray}{rgb}{0.5,0.5,0.5}
\definecolor{lightgray}{rgb}{0.97,0.97,0.97}
\definecolor{mauve}{rgb}{0.58,0,0.82}
\definecolor{sepia}{rgb}{0.58,0.1,0.1}

\newcommand{\keyword}[1]{\underline{\textbf{\textsc{#1}}}}
\newcommand{\keywordb}[1]{\textbf{\textsc{#1}}}
\newcommand{\literal}[1]{{\color{sepia}{\textsc{#1}}}}

\lstdefinelanguage{planning}
{
    keywords={
        Action,
        Init,
        Goal
    },
    morekeywords={[2]},
    morekeywords={[3]
        Precond,
        Effect,
    },
    morekeywords={[4]
        Cargo,
        Plane,
        Airport,
        At,
        In,
        Load,
        Unload,
        Fly},
    comment=[l]{\;},
    sensitive=true,
    extendedchars=true,
    literate={∧}{{\textcolor{mauve}{\ensuremath\wedge}}}1 {¬}{{\textcolor{red}{\ensuremath\lnot}}}1
}[keywords, comments]

\lstset{ %
  backgroundcolor=\color{lightgray},
  basicstyle=\footnotesize,
  breakatwhitespace=false,
  breaklines=true,
  captionpos=t,
  commentstyle=\color{mygreen},
  deletekeywords={...},
  escapeinside={\%*}{*)},
  extendedchars=true,
  frame=none,
  keepspaces=true,
  keywordstyle=\color{blue}\keyword,
  keywordstyle=[3]\color{blue}\keywordb,
  keywordstyle=[4]\literal,
  language=planning,
  morekeywords={*,...},
  numbers=left,
  numbersep=5pt,
  numberstyle=\tiny\color{mygray},
  rulecolor=\color{black},
  showspaces=false,
  showstringspaces=false,
  showtabs=false,
  stringstyle=\color{mymauve},
  tabsize=4,
  title=\lstname
}

\begin{document}

\maketitle

%\begin{abstract}
%\ldots
%\end{abstract}


\section{Problems and optimal plans}
The three planning problems take place in the Air Cargo domain that consists of
the object literals \literal{Cargo}, \literal{Plane} and \literal{Airport},
the propositions \literal{At} and \literal{In} as well as the actions
defined in \cref{lst:aircargo}.

\begin{lstlisting}[caption=Air Cargo Action Schema,label=lst:aircargo]
Action(Load(c, p, a),
    Precond: At(c, a) ∧ At(p, a) 
           ∧ Cargo(c) ∧ Plane(p) ∧ Airport(a)
    Effect: ¬ At(c, a) ∧ In(c, p))
    
Action(Unload(c, p, a),
    Precond: In(c, p) ∧ At(p, a) 
           ∧ Cargo(c) ∧ Plane(p) ∧ Airport(a)
    Effect: At(c, a) ∧ ¬ In(c, p))
    
Action(Fly(p, from, to),
    Precond: At(p, from) 
           ∧ Plane(p) ∧ Airport(from) ∧ Airport(to)
    Effect: ¬ At(p, from) ∧ At(p, to))
\end{lstlisting}

\paragraph{Problem 1:} The first planning problem using two cargo items, planes and airports is defined in \cref{lst:problem1}. An optimal plan to it consists of $6$ actions and is shown in \cref{lst:problem1_plan}. Since the planner does not know about durations,
it cannot reason about the order of \literal{Fly} and \literal{Unload} operations that
are indipendent across airplanes: It might be more efficient to schedule all \literal{Fly}
actions before all \literal{Unload}s, because there is no need to wait for a plane to unload
for another to start flying; however, none of that is encoded in the problem domain.
Metrics for the different planning strategies are shown in \cref{tab:problem1_metrics,fig:problem1_metrics} \vpageref{tab:problem1_metrics,fig:problem1_metrics}.

\begin{lstlisting}[caption=Problem 1 initial state and goal,label=lst:problem1]
Init(At(C1, SFO) ∧ At(C2, JFK) 
   ∧ At(P1, SFO) ∧ At(P2, JFK) 
   ∧ Cargo(C1) ∧ Cargo(C2) 
   ∧ Plane(P1) ∧ Plane(P2)
   ∧ Airport(JFK) ∧ Airport(SFO))
Goal(At(C1, JFK) ∧ At(C2, SFO))
\end{lstlisting}

\begin{lstlisting}[caption=Problem 1 optimal plan,label=lst:problem1_plan]
Load(C1, P1, SFO)
Load(C2, P2, JFK)
Fly(P2, JFK, SFO)
Unload(C2, P2, SFO)
Fly(P1, SFO, JFK)
Unload(C1, P1, JFK)
\end{lstlisting}

\paragraph{Problem 2:} The second planning problem uses three cargo items, planes and airports and is defined in \cref{lst:problem2}. An optimal plan to it consists of $9$ actions and is shown in \cref{lst:problem2_plan}.
Metrics for the different planning strategies are shown in \cref{tab:problem2_metrics,fig:problem2_metrics} \vpageref{tab:problem2_metrics,fig:problem2_metrics}.

\begin{lstlisting}[caption=Problem 2 initial state and goal,label=lst:problem2]
Init(At(C1, SFO) ∧ At(C2, JFK) ∧ At(C3, ATL) 
   ∧ At(P1, SFO) ∧ At(P2, JFK) ∧ At(P3, ATL) 
   ∧ Cargo(C1) ∧ Cargo(C2) ∧ Cargo(C3)
   ∧ Plane(P1) ∧ Plane(P2) ∧ Plane(P3)
   ∧ Airport(JFK) ∧ Airport(SFO) ∧ Airport(ATL))
Goal(At(C1, JFK) ∧ At(C2, SFO) ∧ At(C3, SFO))
\end{lstlisting}

\begin{lstlisting}[caption=Problem 2 optimal plan,label=lst:problem2_plan]
Load(C1, P1, SFO)
Load(C2, P2, JFK)
Load(C3, P3, ATL)
Fly(P2, JFK, SFO)
Unload(C2, P2, SFO)
Fly(P1, SFO, JFK)
Unload(C1, P1, JFK)
Fly(P3, ATL, SFO)
Unload(C3, P3, SFO)
\end{lstlisting}

\paragraph{Problem 3:} The third and last planning problem uses four cargo items and airports with only two planes; it is defined in \cref{lst:problem3}. An optimal plan to it consists of $12$ actions and is shown in \cref{lst:problem3_plan}.
In this plan, plane \textsc{P2} is loaded with both cargo \textsc{C2} and \textsc{C4} at
the airports \textsc{JFK} (\cref{lst:p3_1st_load}) and \textsc{ORD} (\cref{lst:p3_2nd_load}) respectively, and both are then flown to \textsc{SFO} (\cref{lst:p3_fly}).
This is a valid plan, as an airplane does not have to be empty in order to be \literal{Load}ed,
according to the preconditions of that action. If required, the behavior of flying
only single cargos could be achieved by placing a virtual \textsc{Empty} cargo into the plane,
similar to the \textsc{Empty} block of the 8-Puzzle domain.
Metrics for the different planning strategies are shown in \cref{tab:problem3_metrics,fig:problem3_metrics} \vpageref{tab:problem3_metrics,fig:problem3_metrics}.

\begin{lstlisting}[caption=Problem 3 initial state and goal,label=lst:problem3]
Init(At(C1, SFO) ∧ At(C2, JFK) ∧ At(C3, ATL) ∧ At(C4, ORD) 
   ∧ At(P1, SFO) ∧ At(P2, JFK) 
   ∧ Cargo(C1) ∧ Cargo(C2) ∧ Cargo(C3) ∧ Cargo(C4)
   ∧ Plane(P1) ∧ Plane(P2)
   ∧ Airport(JFK) ∧ Airport(SFO) ∧ Airport(ATL) ∧ Airport(ORD))
Goal(At(C1, JFK) ∧ At(C3, JFK) ∧ At(C2, SFO) ∧ At(C4, SFO))
\end{lstlisting}

\begin{lstlisting}[caption=Problem 3 optimal plan,label=lst:problem3_plan]
Load(C1, P1, SFO)
Load(C2, P2, JFK) ; 1st load to P2 %* \label{lst:p3_1st_load} *)
Fly(P2, JFK, ORD) ; P2 carries C2
Load(C4, P2, ORD) ; 2nd load to P2 %* \label{lst:p3_2nd_load} *)
Fly(P1, SFO, ATL)
Load(C3, P1, ATL)
Fly(P1, ATL, JFK)
Unload(C1, P1, JFK)
Unload(C3, P1, JFK)
Fly(P2, ORD, SFO) ; P2 carries C2 and C4 %* \label{lst:p3_fly} *)
Unload(C2, P2, SFO)
Unload(C4, P2, SFO)
\end{lstlisting}

\section{Uninformed and informed planning}

\paragraph{Non-heuristic algorithms:} The non-heuristic strategies used
were \href{https://github.com/sunsided/aima-pseudocode/blob/master/md/Breadth-First-Search.md}{\textit{Breadth-first}} and \href{https://github.com/sunsided/aima-pseudocode/blob/master/md/Tree-Search-and-Graph-Search.md}{\textit{Breadth-first Tree}} (tree search using a FIFO queue-based frontier),
\href{https://github.com/sunsided/aima-pseudocode/blob/master/md/Tree-Search-and-Graph-Search.md}{\textit{Depth-first Graph}} (graph search using a stack-based frontier), \href{https://github.com/sunsided/aima-pseudocode/blob/master/md/Depth-Limited-Search.md}{\textit{Depth-limited}} and \href{https://github.com/sunsided/aima-pseudocode/blob/master/md/Uniform-Cost-Search.md}{\textit{Uniform cost}} (best-first graph using the path cost as the score) search.

Out of these search strategies, the \textit{depth-first graph} search consistently
is among the fastest execution times, but never resulted in
optimal plans; on the contrary, plan lengths resulting from this algorithm
were one to two orders of magnitudes off from the optimal plan length,
often undoing actions performed in previous states in order to then repeat this cycle.
However, the number of node expansions and goal tests was the smallest
within the group of uninformed algorithms and close to the best algorithms
in the informed group.
\textit{Breadth-first} and \textit{uniform cost} searches achieved similar results in
number of espansions, goal tests, node creations and execution time,
yielding optimal plans both on all problems.\footnote{
Since the path cost tends to increase with the depth of the search tree, both
strategies essentially use a similar approach.}
For more complex problems, uniform cost search turned out to be slightly more efficient in execution time despite the slightly bigger amount of node explorations
performed.
The \textit{Breadth-first tree} and \textit{depth limited} searches failed to converge
to a solution within a six hour timeframe for the problems 2 (\cref{tab:problem2_metrics}) and 3 (\cref{tab:problem3_metrics}).
\textit{Depth-limited} search was the worst performer for problem 2 in terms of execution time and the second-worst in problem 1,
directly followed by \textit{Breadth-first tree} search (\cref{tab:problem1_metrics}) .

\paragraph{Heuristic algorithms:} The heuristic strategies used
were \href{https://github.com/sunsided/aima-pseudocode/blob/master/md/Recursive-Best-First-Search.md}{\textit{Recursive best-first}}, \textit{Greedy best-first} (best-first search using the heuristic as the score) and \textit{A*} (best-first with the sum of path cost and heuristic as the score) searches with 
\textit{Constant cost (H1)} (every action has a cost heuristic of $1$), \textit{Precondition ignoring} (every action is applicable) and \textit{\href{https://github.com/sunsided/aima-pseudocode/blob/master/md/GraphPlan.md}{Planning Graph} Level-sum} heuristics.

Out of the informed strategies, \textit{Greedy best-first Graph H1} consistently achieved
the fastest execution time, but failed to yield optimal plans for
the two more complex problems.
The \textit{A*} algorithm with \textit{Planning Graph Level-sum} heuristic,
on the other hand, consistently resulted in the lowest amount of explored
states and resulted in optimal plans, but also required an execution time
that was about tenfold of the others (the exception here is \textit{Recursive best-first Graph} search with \textit{H1} (non-)heuristic which did not finish within a six hour timeframe for problems $2$ and $3$).
The second-best (informed) algorithm across all problems appear \textit(A*) with \textit{ignoring preconditions} heuristic,
which resulted in a significantly higher number of explored nodes and goal tests
but required only a tenth of the execution time of the \textit{Planning Graph Level-sum} variant.

The \textit{Planning Graph Level-sum} heuristic results in the lowest number
of explorations (significantly less than the competitors) 
and should consequently be a very good candidate for the best algorithm.
Since the difference is in the heuristic, the comparatively long execution time 
must come from the planning graph construction and level-sum evaluation. 
Optimizing the implementation might result in a much better rating;
as of writing this document, however, no further optimization was performed.

\paragraph{Performance between classes:} If plan optimality is not required,
\textit{Greedy best-first Graph H1} performed best according to execution time
with number of explored states within one order of magnitude of the best 
algorithm; the plan length consistently was less than twice the length of the
optimal plan for each problem.
When number of explorations and plan optimality is critical, \textit{A*} with \textit{Planning Graph Level-sum} heuristic is the go-to choice, with
execution times within one order of magnitude of the fastest (optimal) algorithm.\footnote{As discussed, an optimized implementation of both the planning graph
and the heuristic might improve the rating in terms of execution time.}
If plan optimality and execution time are favored, \textit{A*} with \textit{Ignore preconditions} is the best candidate across the given problems, yielding optimal
plans and one of the fastest execution times even on the more complex problems.
The \textit{Depth-first Graph} algorithm finally resulted in the fastest execution
times with low explorations but resulted in the longest plans.
It might be worthwile to derive a plan using this algorithm first and then fine-tune
it to obtain an optimal result.

In comparison to the uninformed algorithms, all optimal informed strategies
were faster except for the simplest case of problem 1 (also ignoring \textit{Planning Graph Level-sum} heuristic as discussed and \textit{Recursive best-first} search with the \textit(H1) non-heuristic which failed to converge).
This leads to the conclusion that adding any guides (in the form of heuristics) that help directing the search
in a graph results in faster convergence to the solution,
even in presence of suboptimal heuristics.

\pagebreak

\pgfplotstableset{col sep=comma,
%      fixed,
%      precision=4,
	  column type={>{\fontseries{bx}\selectfont}l},
	  every head row/.style={after row=\hline},
	  postproc cell content/.append style={
/pgfplots/table/@cell content/.add={\fontseries{\seriesdefault}\selectfont}{}},
      columns/Algorithm/.style={string type},
      columns/New Nodes/.style={column name=New nodes},
      columns/Goal Tests/.style={column name=Goal tests},
      columns/Plan length/.style={column name=Length},
      columns={Algorithm,Expansions,Goal Tests,New Nodes,Plan length,Duration (s)},
      postproc cell content/.append style={
/pgfplots/table/@cell content/.add={\fontseries{\seriesdefault}\selectfont}{}}
      }
      
\pgfplotstableread{metrics_1.csv}{\metricsfirst}
\pgfplotstableread{metrics_2.csv}{\metricssecond}
\pgfplotstableread{metrics_3.csv}{\metricsthird}

\begin{table*}[hp]
\pgfplotstabletypeset{metrics_1.csv}
\caption{Metrics for problem 1}
\floatfoot{The optimal plan length for this problem is $6$ actions.}
\label{tab:problem1_metrics}
\end{table*}

\begin{figure}[hp]
  \centering
\begin{tikzpicture}
  \begin{axis}[
	    width=12cm,
	    height=7cm,
  		ybar,ymode=log,
  		bar width=2pt,
  		xtick=data,
  		xticklabels={Breadth-first,Breadth-first Tree,Depth-first Graph,Depth limited,Uniform cost,Recursive best-first H1,Greedy best-first Graph H1,A* H1,A* Ignore Preconditions,A* PG Levelsum},
  		x tick label style={rotate=45,anchor=east,font=\footnotesize},
  		legend pos=outer north east,
  		legend cell align={left},
  		legend style={
  			anchor=north west}]
    \addplot table[x=Metric, y=Expansions] {\metricsfirst};
    \addplot table[x=Metric, y=Goal Tests] {\metricsfirst};
    \addplot table[x=Metric, y=New Nodes] {\metricsfirst};
    \addplot table[x=Metric, y=Duration (s)] {\metricsfirst}; 
    
    \addlegendentry{Expansions (count)}
    \addlegendentry{Goal Tests (count)}
    \addlegendentry{New Nodes (count)}
    \addlegendentry{Execution time (seconds)}
  \end{axis}
\end{tikzpicture}
\caption{Metrics for problem 1}
\label{fig:problem1_metrics}
\end{figure}

\FloatBarrier

\begin{table*}[!hp]
\pgfplotstabletypeset{metrics_2.csv}
\caption{Metrics for problem 2}
\floatfoot{The optimal plan length for this problem is $9$ actions.Two of the ten heuristics could not be measured,
because the execution time of the planner exceeded six hours.}
\label{tab:problem2_metrics}
\end{table*}

\begin{figure}[!hp]
  \centering
\begin{tikzpicture}
  \begin{axis}[
	    width=12cm,
	    height=7cm,
  		ybar,ymode=log,
  		bar width=2pt,
  		xtick=data,
  		xticklabels={Breadth-first,Depth-first Graph,Depth limited,Uniform cost,Greedy best-first Graph H1,A* H1,A* Ignore Preconditions,A* PG Levelsum},
  		x tick label style={rotate=45,anchor=east,font=\footnotesize},
  		legend pos=outer north east,
  		legend cell align={left},
  		legend style={
  			anchor=north west}]
    \addplot table[x=Metric, y=Expansions] {\metricssecond};
    \addplot table[x=Metric, y=Goal Tests] {\metricssecond};
    \addplot table[x=Metric, y=New Nodes] {\metricssecond};
    \addplot table[x=Metric, y=Duration (s)] {\metricssecond}; 
    
    \addlegendentry{Expansions (count)}
    \addlegendentry{Goal Tests (count)}
    \addlegendentry{New Nodes (count)}
    \addlegendentry{Execution time (seconds)}
  \end{axis}
\end{tikzpicture}
\caption{Metrics for problem 2}
\label{fig:problem2_metrics}
\end{figure}

\FloatBarrier

\begin{table*}[!hp]
\centering
\pgfplotstabletypeset{metrics_3.csv}
\caption{Metrics for problem 3}
\floatfoot{The optimal plan length for this problem is $12$ actions.
Three of the ten heuristics could not be measured,
because the execution time of the planner exceeded six hours.}
\label{tab:problem3_metrics}
\end{table*}

\begin{figure}[!hp]
  \centering
\begin{tikzpicture}
  \begin{axis}[
	    width=12cm,
	    height=7cm,
  		ybar,ymode=log,
  		bar width=2pt,
  		xtick=data,
  		xticklabels={Breadth-first,Depth-first Graph,Uniform cost,Greedy best-first Graph H1,A* H1,A* Ignore Preconditions,A* PG Levelsum},
  		x tick label style={rotate=45,anchor=east,font=\footnotesize},
  		legend pos=outer north east,
  		legend cell align={left},
  		legend style={
  			anchor=north west}]
    \addplot table[x=Metric, y=Expansions] {\metricsthird};
    \addplot table[x=Metric, y=Goal Tests] {\metricsthird};
    \addplot table[x=Metric, y=New Nodes] {\metricsthird};
    \addplot table[x=Metric, y=Duration (s)] {\metricsthird}; 
    
    \addlegendentry{Expansions (count)}
    \addlegendentry{Goal Tests (count)}
    \addlegendentry{New Nodes (count)}
    \addlegendentry{Execution time (seconds)}
  \end{axis}
\end{tikzpicture}
\caption{Metrics for problem 3}
\label{fig:problem3_metrics}
\end{figure}

\end{document}
